\#\-Quadratic Sieve

\subsection*{Introduction}

The quadratic sieve algorithm (Q\-S) is an integer factorization algorithm and, in practice, the second fastest method known (after the general number field sieve). It is still the fastest for integers under 100 decimal digits or so, and is considerably simpler than the number field sieve. It is a general-\/purpose factorization algorithm, meaning that its running time depends solely on the size of the integer to be factored, and not on special structure or properties. It was invented by Carl Pomerance in 1981 as an improvement to Schroeppel's linear sieve (from Wikipedia)

Our version of the algorithm is a parallel Object Oriented implementation, written in C++.

\subsection*{Usage}

This version of the algorithm is command-\/line software, the parameter are specified when Q\-S is launched. The parameters that can be specified are\-:
\begin{DoxyItemize}
\item Number to be factored
\item Number of Processors to use
\item Number of Cores to use
\item Base Factor
\item Range
\item Polynomial to use for the factorization
\end{DoxyItemize}

\subsection*{License}

Quadratic Sieve is free software\-: you can redistribute it and/or modify it under the terms of the G\-N\-U General Public License as published by the Free Software Foundation, either version 3 of the License, or (at your option) any later version. Quadratic Sieve is distributed in the hope that it will be useful, but W\-I\-T\-H\-O\-U\-T A\-N\-Y W\-A\-R\-R\-A\-N\-T\-Y; without even the implied warranty of M\-E\-R\-C\-H\-A\-N\-T\-A\-B\-I\-L\-I\-T\-Y or F\-I\-T\-N\-E\-S\-S F\-O\-R A P\-A\-R\-T\-I\-C\-U\-L\-A\-R P\-U\-R\-P\-O\-S\-E. See the G\-N\-U General Public License for more details. You should have received a copy of the G\-N\-U General Public License along with Quadratic Sieve. If not, see \href{http://www.gnu.org/licenses/}{\tt http\-://www.\-gnu.\-org/licenses/}.

\subsection*{Authors}

Ayoub Ouarrak, Lorenzo Pattarini 